\documentclass{article}
\usepackage[utf8]{inputenc}
\usepackage{blindtext}
\usepackage{gensymb}
\usepackage{tabularx}
\usepackage{array}
\usepackage{graphicx}
\usepackage{amsmath}
\usepackage{amssymb}
%\usepackage{unicode-math}
\usepackage{chngpage}
\usepackage{multirow}  
\usepackage{varwidth}
\usepackage{pdfpages}
\usepackage{float}
\usepackage{wrapfig}
\usepackage{polski}
\usepackage{caption}
\usepackage{subcaption}
\usepackage[export]{adjustbox}
\usepackage[font=small,labelfont=bf]{caption}
\usepackage{makecell}
\usepackage{multirow}
\usepackage[margin=1in]{geometry}
\usepackage{listings}


\renewcommand\theadalign{bc}
\renewcommand\theadfont{\bfseries}
\renewcommand\theadgape{\Gape[4pt]}
\renewcommand\cellgape{\Gape[4pt]}



\begin{document}
 \begin{center}
  \includegraphics[width=0.8\textwidth]{logo_pwr.png} \\[1cm] % Logo PWr
        %\vfill
        \bigskip
            \begin{tabular}{|c|c|c|c|}
                \hline \multicolumn{3}{|c|}{\thead{ Metody numeryczne i optymalizacja}} \\
                \hline \makecell{Termin:\\piątek: 11:15} & \makecell{Nr ćwiczenia:\\1} & \makecell
                {METODY BEZPOŚREDNIE DLA ROZWIĄZYWANIA \\
UKŁADÓW RÓWNAŃ LINIOWYCH}  \\
                \hline \makecell{Nr grupy: \\ 6} & \makecell{Data wykonania\\ ćwiczenia:\\14.03.25} & \makecell{Skład grupy:\\{Mateusz Wilk 268558}\\{Mateusz Kowalczyk 268533}} \\
                \hline
            \end{tabular}
            %\vfill
          \end{center}

\newpage

\section{Wstęp}
Rozwiązywanie układów równań liniowych odgrywa kluczową rolę w wielu dziedzinach nauki i techniki, takich jak inżynieria, ekonomia, fizyka czy informatyka. W praktyce stosuje się różne metody numeryczne umożliwiające znalezienie rozwiązania tych układów, w tym metody bezpośrednie oraz iteracyjne.

Metody bezpośrednie, takie jak eliminacja Gaussa czy faktoryzacja macierzy, pozwalają uzyskać dokładne rozwiązanie w skończonej liczbie kroków, co czyni je szczególnie przydatnymi w przypadku dobrze uwarunkowanych układów. W rzeczywistych obliczeniach komputerowych istotnym problemem są jednak błędy numeryczne wynikające z ograniczonej precyzji reprezentacji liczb w systemach komputerowych. Błędy te mogą prowadzić do propagacji niedokładności w kolejnych etapach obliczeń, co wpływa na stabilność i dokładność uzyskanego rozwiązania.

Analiza tych błędów oraz wpływu warunkowania macierzy na stabilność obliczeń ma kluczowe znaczenie w ocenie efektywności stosowanych metod. W ramach niniejszego ćwiczenia skupiono się na metodach bezpośrednich, które pozwalają uzyskać dokładne rozwiązanie w skończonej liczbie kroków, o ile nie występują istotne błędy numeryczne.
\section{Cel ćwiczenia}
Celem ćwiczenia jest zapoznanie się z metodami bezpośrednimi rozwiązywania układów równań liniowych, w tym eliminacją Gaussa i faktoryzacją macierzy (LU, QR). Ćwiczenie obejmuje analizę wybranych algorytmów, ich implementację w środowisku programistycznym Python, bądź ich ręczne wykonanie w zależności od wymagań przedstawionych w treści polecenia. Dodatkowo ćwiczenie obejmuje również analizę wpływu błędów numerycznych i warunkowania macierzy na dokładność wykonywanych obliczeń.

\section{Zadania}
\subsection{Zadanie 1.}

\[
\begin{cases}
    2u - v = 0, \\
    -u + 2v - w = 0, \\
    -v + 2w - z = 0, \\
    -w + 2z = 5.
\end{cases}
\]
\newline
Przedstawiamy układ równań w postaci macierzowej:

\[
\begin{bmatrix}
2 & -1 & 0 & 0 & | 0 \\
-1 & 2 & -1 & 0 & | 0 \\
0 & -1 & 2 & -1 & | 0 \\
0 & 0 & -1 & 2 & | 5
\end{bmatrix}
\]

\textbf{Krok 1:} Wybieramy pierwszy element jako pivot (2 w pierwszym wierszu) i eliminujemy pierwszą kolumnę:

Dodajemy pierwszy wiersz pomnożony przez \( \frac{1}{2} \) do drugiego wiersza:

\[
R_2 = R_2 + \frac{1}{2} R_1
\]

Nowa macierz:

\[
\begin{bmatrix}
2 & -1 & 0 & 0 & | 0 \\
0 & \frac{3}{2} & -1 & 0 & | 0 \\
0 & -1 & 2 & -1 & | 0 \\
0 & 0 & -1 & 2 & | 5
\end{bmatrix}
\]

\textbf{Krok 2:} Wybieramy drugi element (1.5) jako pivot i eliminujemy drugą kolumnę:

Dodajemy drugi wiersz pomnożony przez \( \frac{2}{3} \) do trzeciego wiersza:

\[
R_3 = R_3 + \frac{2}{3} R_2
\]

Nowa macierz:

\[
\begin{bmatrix}
2 & -1 & 0 & 0 & | 0 \\
0 & \frac{3}{2} & -1 & 0 & | 0 \\
0 & 0 & \frac{4}{3} & -1 & | 0 \\
0 & 0 & -1 & 2 & | 5
\end{bmatrix}
\]

\textbf{Krok 3:} Wybieramy trzeci element \( \frac{4}{3} \) jako pivot i eliminujemy trzecią kolumnę:

Dodajemy trzeci wiersz pomnożony przez \( \frac{3}{4} \) do czwartego wiersza:

\[
R_4 = R_4 + \frac{3}{4} R_3
\]

Nowa macierz:

\[
\begin{bmatrix}
2 & -1 & 0 & 0 & | 0 \\
0 & 1.5 & -1 & 0 & | 0 \\
0 & 0 & \frac{4}{3} & -1 & | 0 \\
0 & 0 & 0 & \frac{5}{4} & | 5
\end{bmatrix}
\]

\textbf{Krok 4:} Rozwiązujemy układ równań wstecz.

Z ostatniego równania:

\[
\frac{5}{4}z = 5 \Rightarrow z\frac{5}{\frac{5}{4}} = 4
\]

Z trzeciego równania:

\[
\frac{4}{3}w - 4 = 0 \Rightarrow w = \frac{4}{\frac{4}{3}} = 3
\]

Z drugiego równania:

\[
\frac{3}{2}v - 3 = 0 \Rightarrow v = \frac{3}{\frac{3}{2}} = 2
\]

Z pierwszego równania:

\[
2u - 2 = 0 \Rightarrow 2u = 2 = 1
\]

\textbf{Ostateczne rozwiązanie:}

\[
u = 4, \quad v = 3, \quad w = 2, \quad z = 1
\]

Elementy podstawowe (pivots): \( 2, \frac{3}{2}, \frac{4}{3}, \frac{5}{4} \).
\subsection{Zadanie 2.}
\[
\begin{cases}
x_1 + x_2 + x_3 = 1, \\
x_1 + x_2 + 2x_3 = 2, \\
x_1 + 2x_2 + 2x_3 = 1.
\end{cases}
\]
\textbf{Krok 1:} Zapisujemy układ równań w postaci macierzowej:

\[
\begin{bmatrix}
1 & 1 & 1 & | 1 \\
1 & 1 & 2 & | 2 \\
1 & 2 & 2 & | 1
\end{bmatrix}
\]

Pierwszy element (1 w lewym górnym rogu) jest pivotem.

\textbf{Krok 2:} Eliminacja pierwszej kolumny.

Odejmujemy pierwszy wiersz od drugiego:

\[
R_2 = R_2 - R_1
\]

Odejmujemy pierwszy wiersz od trzeciego:

\[
R_3 = R_3 - R_1
\]

Nowa macierz:

\[
\begin{bmatrix}
1 & 1 & 1 & | 1 \\
0 & 0 & 1 & | 1 \\
0 & 1 & 1 & | 0
\end{bmatrix}
\]

\textbf{Krok 3:} Wybór nowego pivota. Drugi wiersz ma element zerowy na diagonali, więc musimy zamienić wiersze 2 i 3:

\[
\begin{bmatrix}
1 & 1 & 1 & | 1 \\
0 & 1 & 1 & | 0 \\
0 & 0 & 1 & | 1
\end{bmatrix}
\]

Teraz mamy poprawne pivoty.

\textbf{Krok 4:} Eliminacja trzeciej kolumny.

Odejmujemy drugi wiersz od trzeciego:

\[
R_3 = R_3 - R_2
\]

Ale trzeci wiersz już jest w postaci schodkowej, więc możemy przejść do rozwiązania układu:

\textbf{Krok 5:} Rozwiązanie układu równań wstecz.

Z trzeciego równania:

\[
x_3 = 1
\]

Z drugiego równania:

\[
x_2 + x_3 = 0 \Rightarrow x_2 = -1
\]

Z pierwszego równania:

\[
x_1 + x_2 + x_3 = 1 \Rightarrow x_1 - 1 + 1 = 1 \Rightarrow x_1 = 1
\]

\textbf{Ostateczne rozwiązanie:}

\[
x_1 = 1, \quad x_2 = -1, \quad x_3 = 1
\]

\textbf{Dlaczego eliminacja Gaussa bez wyboru elementu podstawowego nie działa poprawnie?}

Bez częściowego wyboru elementu podstawowego moglibyśmy znaleźć się w sytuacji, w której pivot (element diagonalny) byłby zerem, co uniemożliwiłoby poprawne wykonanie eliminacji. W tym przypadku, gdybyśmy nie zamienili drugiego i trzeciego wiersza, to eliminacja prowadziłaby do błędnych obliczeń lub konieczności dzielenia przez zero.

\subsection{Zadanie 3.}

\[
\begin{cases}
0.0001x_1 + x_2 = 1, \\
x_1 + x_2 = 2.
\end{cases}
\]

\textbf{Rozwiązanie bez wyboru elementu podstawowego:}

Przekształcamy układ do postaci macierzowej:

\[
\begin{bmatrix}
0.0001 & 1 & | 1 \\
1 & 1 & | 2
\end{bmatrix}
\]

Wybieramy \( 0.0001 \) jako pierwszy pivot i eliminujemy pierwszą kolumnę:

\[
R_2 = R_2 - 10^4 R_1
\]

\[
\begin{bmatrix}
0.0001 & 1 & | 1 \\
0 & -9999 & | -9998
\end{bmatrix}
\]

\textbf{Krok 2:} Rozwiązujemy układ wstecz.

Z drugiego równania:

\[
-999 x_2 = -9998 \Rightarrow x_2 \approx 1
\]

Podstawiamy do pierwszego równania:

\[
0.0001 x_1 + 0.999 = 1 \Rightarrow x_1 = \frac{0.999}{0.0001} = 10
\]


\textbf{Rozwiązanie:}

\[
x_1 = 10, \quad x_2 = 1
\]

\textbf{Rozwiązanie z wyborem elementu podstawowego:}

Zamieniamy wiersze, aby największy element w pierwszej kolumnie był pivotem:

\[
\begin{bmatrix}
1 & 1 & | 2 \\
0.0001 & 1 & | 1
\end{bmatrix}
\]

Eliminujemy pierwszy element w drugim wierszu:

\[
R_2 = R_2 - 0.0001 R_1
\]

\[
\begin{bmatrix}
1 & 1 & | 2 \\
0 & 0.999 & | 0.999
\end{bmatrix}
\]

Z drugiego równania:

\[
0.999 x_2 = 0.999
\]

\[
x_2 = 1
\]

Podstawiamy do pierwszego równania:

\[
x_1 + 1 = 2
\]

\[
x_1 = 1
\]

\textbf{Ostateczne rozwiązanie:}

\[
x_1 = 1.00, \quad x_2 = 1.00
\]

\textbf{Wniosek:} Bez wyboru elementu podstawowego uzyskaliśmy dużą wartość \( x_1 \), co wskazuje na błąd numeryczny. Z wyborem pivota wynik jest poprawny.

\subsection{Zadanie 4.}
\textbf{Obliczanie układu rownań metodą eliminacji Gaussa.}
\begin{lstlisting}
    import numpy as np

# Eliminacja Gaussa:
def gaussian_elimination(A, b):
    n = len(b)              # warunek spelniony zawsze dla rownan liniowych 
    A = A.astype(float)     # konwersja na typ zmiennoprzecinkowy
    b = b.astype(float)

    # Eliminacja wierszy
    for i in range(n):                  # przechodzimy przez kazdy wiersz macierzy              
        for j in range(i + 1, n):       # iteracja przez wiersze ponizej aktualnego pivotu
            factor = A[j, i] / A[i, i]  
            for k in range(i, n):       
                A[j, k] -= factor * A[i, k] # odejmujemy wielkorotnosc pivotu przez factor aby uzyskac zero 
            b[j] -= factor * b[i]          # analogicznie dla wyrazow wolnych 

    # Podstawianie wsteczne:
    x = np.zeros(n)                         
    for i in range(n - 1, -1, -1):          
        sum_ax = 0
        for j in range(i + 1, n):           
            sum_ax += A[i, j] * x[j]
        x[i] = (b[i] - sum_ax) / A[i, i]    

    return x
\end{lstlisting}

\begin{lstlisting}
    
# Norma macierzy:
def matrix_norm_1(A):
    return max(np.sum(np.abs(A), axis=0))  
        return norm_A * norm_A_inv
\end{lstlisting}
    Oblicza norme macierzy $||A||_1$ jako maksimum sumy wartości bezwzględnych w kolumnach.
\begin{lstlisting}
# Wskaznik uwarunkowania macierzy:
def condition_number_1(A):
    A_inv = np.linalg.inv(A)  
    norm_A = matrix_norm_1(A)  
    norm_A_inv = matrix_norm_1(A_inv)  

    return norm_A * norm_A_inv
\end{lstlisting}

Oblicza wskaznik uwarunkowania macierzy jako $||A||_1 * ||A^{(-1)}||_1$.

\begin{figure}[h]
    \centering
    \includegraphics[width=0.5\linewidth]{wynikipolecenia/wynikizad4.png}
    \caption{Wynik zadania 4.}
    \label{zad4wyniki}
\end{figure}

\subsubsection{Wnioski}
Macierz A jest żle uwarunkowana (wartość uwarunkowania macierzy powinna być jak najbliżej 1), co oznacza, że układ równań jest bardzo wrażliwy na błędy zaokrągleń i drobne zaburzenia danych.To prowadzi do niestabilności w rozwiązaniach, co widzimy na przykłądzie - minimalna zmiana b2 spowodowała ogromny skok w wartościach x1 i x2.


\subsection{Zadanie 5.}
\begin{lstlisting}
import numpy as np

def lu_decomposition(A):
    """
    Faktoryzacja LU macierzy A, zwraca macierze L i U
    """
    n = len(A)
    L = np.eye(n)           # utworzenie macierzy jednostkowej dolnotrojkatnej L
    U = A.astype(float)     # przypisanie wartosci macierzy A do macierzy U 
                            (docelowo gornotrojkatnej)

    for i in range(n):
        for j in range(i+1, n):
            factor = U[j, i] / U[i,i]       # obliczanie mnoznika
            L[j,i] = factor                 # zapamietanie mnoznika w macierzy L
            for k in range(i, n):
                U[j,k] -= factor * U[i, k]  # eliminacja dolnej czesci macierzy

    return L, U

def forward_substitution(L, b):
    """
    Rozwiazanie ukladu Ly = b (podstawianie do przodu).
    """
    n = len(b)
    y = np.zeros(n)

    for i in range(n):
        sum_Ly = sum(L[i, j] * y[j] for j in range(i))
        y[i] = (b[i] - sum_Ly) / L[i, i]

    return y


def backward_substitution(U, y):
    """
    Rozwiazanie ukladu Ux = y (podstawianie wsteczne).
    """
    n = len(y)
    x = np.zeros(n)

    for i in range(n - 1, -1, -1):
        sum_Ux = sum(U[i, j] * x[j] for j in range(i + 1, n))
        x[i] = (y[i] - sum_Ux) / U[i, i]

    return x


# Nasze dane z zadania:
A = np.array([[1, 2, 3, 4],
              [-1, 1, 2, 1],
              [0, 2, 1, 3],
              [0, 0, 1, 1]])

b = np.array([1, 1, 1, 1])

# Faktoryzacja:
L, U = lu_decomposition(A)

# Rozwiazanie ukladu rownan:
y = forward_substitution(L, b)
x = backward_substitution(U, y)

# Wyznacznik macierzy:
det_A = np.prod(np.diag(U))     # oblicza iloczyn wszystkich elementow w podanej tablicy

# Wyniki:
print("Macierz L:\n", L, "\n")
print("Macierz U:\n", U, "\n")
print(f"Wyznacznik macierzy A: {det_A}\n")
print(f"Rozwiazanie ukladu Ax = b: x = {[np.round(x,3).tolist()]}\n") # zaokraglenie do 3 
                                                                        liczb znaczacych
\end{lstlisting}
\begin{figure}[h]
    \centering
    \includegraphics[width=0.5\linewidth]{wynikipolecenia/zad5.png}
    \caption{Wyniki faktoryzacji LU}
    \label{zad8wyniki}
\end{figure}

\subsection{Zadanie 6.}

Dana jest macierz:

\[
A =
\begin{bmatrix}
1 & 2 & 2 & 3 & 1 \\
2 & 4 & 4 & 6 & 2 \\
3 & 6 & 6 & 9 & 6 \\
1 & 2 & 4 & 5 & 3
\end{bmatrix}
\]

Naszym celem jest przekształcenie jej do postaci \textbf{RREF (Reduced Row Echelon Form)}, wyznaczenie $\text{rank}(A)$ oraz wskazanie zmiennych podstawowych i wolnych.

\textbf{Krok 1: Zerowanie pierwszej kolumny}

Pierwszy element wiodący to $1$ w pierwszym wierszu. Zerujemy pierwszą kolumnę poniżej niego:

\[
R_2 = R_2 - 2R_1
\]
\[
R_3 = R_3 - 3R_1
\]
\[
R_4 = R_4 - R_1
\]

Po wykonaniu operacji macierz przyjmuje postać:

\[
A =
\begin{bmatrix}
1 & 2 & 2 & 3 & 1 \\
0 & 0 & 0 & 0 & 0 \\
0 & 0 & 0 & 0 & 3 \\
0 & 0 & 2 & -1 & 1
\end{bmatrix}
\]

\textbf{Krok 2: Wybór kolejnego elementu wiodącego}

Pomijamy drugi wiersz, ponieważ składa się z samych zer. Przechodzimy do trzeciego wiersza i normalizujemy trzecią kolumnę.

Nowy element wiodący to $2$ w czwartym wierszu i trzeciej kolumnie. Dzielimy czwarty wiersz przez $2$, aby uzyskać jedynkę:

\[
R_4 = \frac{1}{2} R_4
\]

\[
A =
\begin{bmatrix}
1 & 2 & 2 & 3 & 1 \\
0 & 0 & 2 & -1 & 1 \\
0 & 0 & 0 & 0 & 3 \\
0 & 0 & 0 & 0 & 0
\end{bmatrix}
\]

Zerujemy wartości powyżej nowego elementu wiodącego:

\[
R_3 = R_3 - 3R_4
\]

Po tej operacji macierz przyjmuje postać:

\[
A =
\begin{bmatrix}
1 & 2 & 2 & 3 & 1 \\
0 & 0 & 1 & - \frac{1}{2} & \frac{1}{2} \\
0 & 0 & 0 & 0 & 1 \\
0 & 0 & 0 & 0 & 0
\end{bmatrix}
\]

\textbf{Krok 3: Ostateczna postać RREF}

Usuwamy zerowy wiersz:

\[
A =
\begin{bmatrix}
1 & 2 & 2 & 3 & 1 \\
0 & 0 & 1 & - \frac{1}{2} & \frac{1}{2} \\
0 & 0 & 0 & 0 & 1 \\
\end{bmatrix}
\]

\textbf{Krok 4: Obliczenie rzędu macierzy}

Rząd macierzy to liczba niezerowych wierszy w macierzy RREF:

\[
\text{rank}(A) = 3
\]

\textbf{Krok 5: Identyfikacja zmiennych podstawowych i wolnych}

Kolumny podstawowe to te, w których znajduje się element wiodący (pivot). Są to kolumny 1, 3, 5.\\
Kolumny wolne to te, w których nie ma elementu wiodącego. Są to kolumny 2 i 4.

\textbf{Ostateczna odpowiedź}

\begin{itemize}
    \item Macierz w postaci RREF:
    \[
\begin{bmatrix}
1 & 2 & 2 & 3 & 1 \\
0 & 0 & 1 & - \frac{1}{2} & \frac{1}{2} \\
0 & 0 & 0 & 0 & 1 \\
\end{bmatrix}
\]
    \item Rząd macierzy:
    \[
    \text{rank}(A) = 3
    \]
    \item Kolumny podstawowe (zmienne bazowe): 1, 3, 5
    \item Kolumny wolne (zmienne swobodne): 2, 4
\end{itemize}

\subsection{Zadanie 7.}

\begin{lstlisting}

import numpy as np


def gram_schmidt(A):
    """Implementacja ortogonalizacji Grama-Schmidta dla macierzy A."""
    m, n = A.shape          # Pobranie wartosci macierzy A (m - wiersze, n - kolumny)
    Q = np.zeros((m, n))    # Tworzymy pusta macierz Q
    R = np.zeros((n, n))    # Tworzymy pusta macierz kwadratowa

    for i in range(n):
        # Pobieramy i-ta kolumne macierzy
        v = A[:, i]

        for j in range(i):
            # Obliczamy rzut wektora a_i na poprzednie q_j
            R[j, i] = np.dot(Q[:, j], A[:, i])   # Iloczyn skalarny q_j i a_i
            v = v - R[j, i] * Q[:, j]            # Usuwamy skladowa w kierunku q_j

        # Normalizacja wektora
        R[i, i] = np.linalg.norm(v)
        Q[:, i] = v / R[i, i]

    return Q, R


# Definiujemy macierz A
A = np.array([
    [-2, 1, 2, 1],
    [2, -1, 2, 1],
    [2, 3, -4, 5],
    [2, 3, 0, -1]
], dtype=float)

# Wyznaczamy faktoryzacje QR
Q, R = gram_schmidt(A)

# Zaokraglenie wyniku:
Q = np.round(Q, 3)  # Zaokraglenie do 3 miejsc po przecinku
R = np.round(R, 3)  # Zaokraglenie macierzy R

# Normalizacja znaku macierzy R
for i in range(R.shape[0]):
    if R[i, i] < 0:     # Jesli wartosc na diagonali jest ujemna
        Q[:, i] *= -1   # Zamiana znaku kolumny Q
        R[i, :] *= -1   # Zamiana znaku wiersza R

# Wyswietlenie wynikow
print("Macierz Q:")
print(Q)

print("\nMacierz R:")
print(R)

\end{lstlisting}

\begin{figure}[h]
    \centering
    \includegraphics[width=0.3\linewidth]{wynikipolecenia/zad7.png}
    \caption{Wyniki faktoryzacji QR}
    \label{zad8wyniki}
\end{figure}


\subsection{Zadanie 8.}
\begin{lstlisting}
import numpy as np
import time
import scipy.linalg as la
import pandas as pd

# Definicja parametrow
M = 200   # Liczba wierszy macierzy C
N = 30   # Rozmiar macierzy jednostkowej
L = 20   # Liczba kolumn macierzy C

# Generowanie macierzy
I_N = np.eye(N)  # Macierz jednostkowa N x N
C = np.random.rand(M, L)  # Macierz losowa M x L
A = np.kron(I_N, C.T @ C)  # Macierz A = I_N \otimes (C^T C)

# Generowanie wektora x i obliczenie prawej strony b = Ax
x = np.random.randn(N * L)  # Wektor losowy o rozkladzie normalnym
b = A @ x  # Wektor prawej strony

# Slownik do przechowywania wynikow
results = {}

# 1. Eliminacja Gaussa
start_time = time.time()
x_gauss = np.linalg.solve(A, b)
gauss_time = time.time() - start_time
gauss_error = np.linalg.norm(x_gauss - x)
results["Gauss"] = (round(gauss_time, 4), format(gauss_error, ".10e"))

# 2. Faktoryzacja LU
start_time = time.time()
lu, piv = la.lu_factor(A)
x_lu = la.lu_solve((lu, piv), b)
lu_time = time.time() - start_time
lu_error = np.linalg.norm(x_lu - x)
results["LU"] = (round(lu_time, 4), format(lu_error, ".10e"))

# 3. Faktoryzacja QR
start_time = time.time()
Q, R = np.linalg.qr(A)
x_qr = la.solve_triangular(R, Q.T @ b)
qr_time = time.time() - start_time
qr_error = np.linalg.norm(x_qr - x)
results["QR"] = (round(qr_time, 4), format(qr_error, ".10e"))

# Tworzenie tabeli wynikow
df_results = pd.DataFrame(results, index=["Czas wykonania (s)", "Blad aproksymacji"])

# Wyswietlenie wynikow
print("\nPorownanie metod rozwiazania ukladu rownan:")
print(df_results)
\end{lstlisting}
\begin{figure}[h]
    \centering
    \includegraphics[width=0.5\linewidth]{wynikipolecenia/wynikizad8.png}
    \caption{Caption}
    \label{zad8wyniki}
\end{figure}
\subsubsection{Wnioski}
Algorytm Gaussa jest najprostszym i najszybszym algorytmem, jednakże nie jest stabilny numerycznie dla źle uwarunkowanych macierzy. Błąd aproksymacji w tym przypadku jest bardzo niewielki.\\
Faktoryzacja LU jest bardziej uniwersalną metodą niż eliminacja Gaussa, ze względu na większą stabilność numeryczną. Jednakże dla naszego przykładu, jest wolniejszą metodą. Błąd aproksymacji jest zbliżony do algorytmu Gaussa. Metoda ta wymaga więcej pamięci do przechowywania macierzy L oraz U. \\
Faktoryzacja QR jest najwolniejszą metodą i charakteryzuje się największym błędem aproksymacji. Podobnie jak faktoryzacja LU, wymaga większej ilości pamięci do przechowywania macierzy Q i R. Mimo to, jest to najbardziej stabilna numerycznie metoda, szczególnie w przypadku źle uwarunkowanych macierzy.


\subsection{Zadanie 9.}
\begin{figure}[h]
    \centering
    \includegraphics[width=0.5\linewidth]{wynikipolecenia/poleceniezad9.png}
    \caption{Obwód elektryczny. Rezystancje wynoszą 20 Ohm}
    \label{zad9polecenie}
\end{figure}
Równania:

\[
\begin{cases}
10V = R I_5 - R I_6 \\
10V = -R I_3 + R I_4 \\
20V = R I_3 + R I_5 \\
0 = I_5 - I_3 - I_2 \\
0 = I_6 + I_5 - I_1 \\
0 = -I_1 + I_3 + I_4
\end{cases}
\]

Dodatkowe zależności:

\[
I_2 = I_5 - I_3
\]

\[
I_1 = I_6 + I_5
\]

\[
I_1 = I_3 + I_4
\]

Macierze:

\[
A =
\begin{bmatrix}
0 & 0 & 0 & 0 & 20 & 20 \\
0 & 0 & -20 & 20 & 0 & 0 \\
0 & 0 & 20 & 0 & 20 & 0 \\
0 & -1 & -1 & 0 & 1 & 0 \\
-1 & 0 & 0 & 0 & 1 & 1 \\
-1 & 0 & 1 & 1 & 0 & 0
\end{bmatrix}
\]

\[
B =
\begin{bmatrix}
10 \\
10 \\
20 \\
0 \\
0 \\
0
\end{bmatrix}
\]
\newpage
\begin{lstlisting}
import numpy as np

# Definicja macierzy A
A = np.array([
    [0, 0, 0, 0, 20, -20],
    [0, 0, -20, 20, 0, 0],
    [0, 0, 20, 0, 20, 0],
    [0, -1, -1, 0, 1, 0],
    [-1, 0, 0, 0, 1, 1],
    [-1, 0, 1, 1, 0, 0]
], dtype=float)

# Definicja macierzy B
B = np.array([10, 10, 20, 0, 0, 0], dtype=float)

# Rozwiazanie ukladu rownan
solution = np.linalg.solve(A, B)

# Wyswietlenie wynikow
print("Rozwiazanie ukladu rownan:")
print(solution)
\end{lstlisting}

\begin{figure}[h]
    \centering
    \includegraphics[width=0.5\linewidth]{wynikipolecenia/wynikzad9.png}
    \caption{Wyniki zadania 9.}
    \label{zad9wyniki}
\end{figure}
\end{document}